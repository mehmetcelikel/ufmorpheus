\section{Introduction}

% Outline 
% ---------------------

% 1. Motivation 
% 2. What are we proposing?  
% 3. Challenges
% 		3.1 User query processing 
% 		3.1 Need of an ontology 
% 4. Realm based question answering 
% 5. About the paper structure 

When traveling though a jungle to a destination, it is easy to get
lost.  The first person to go somewhere may make a number of mistakes
in trying to find the best path to their destination. Those
who come later find it easier to reach the destination if a
well-marked trail has been created. Olsen and Malizia describe this
idea as \emph{exploiting trails} \cite{5379671}.  Rather than treating
a user's discovery experience as a unique entity, one can exploit the fact that a similar search may have already been
performed.  In one study, almost 40 percent of all queries
were repetitions of previous queries\cite{1277770}. Thus, reuse of prior searches is one way to optimize the search
process.  Morpheus is a question answering system motivated by reuse of prior
web search pathways to yield an answer to a user query. Morpheus follows path-finders to their destinations and not only marks the
trail, but also provides a taxi service to take followers to similar
destinations.

There are two distinct Morpheus user roles. A
\textit{path-finder} enters queries in the Morpheus web interface and
searches for an answer to the query using an instrumented web browser. 
This web tracking tool stores the query
and necessary information to revisit the pathways to the page where the path-finder 
found the answer. A \textit{path-follower} uses the Morpheus system much like a regular search engine with a natural language interface. The path-follower enters a question in a text box and receives a guided path to the answer. The system exploits previously found paths to provide an answer.

Morpheus represents user-query details in a semi-structured query (SSQ). It includes a user query in natural-language format, relevant terms for the query, and place holders for the term \textit{categories}. A \textit{category} represents a
topic or class to which a term belongs. For example, the term \textit{tire} can be classified into the category \textit{Automotive Part}.

Morpheus assumes the term categories are classes of a
\textit{consistent} realm-based ontology, that is, one having a singly rooted heterachy whose
subclass/superclass relations have meaningful semantic interpretations. When a path-follower enters a query, Morpheus ranks SSQs in the store based on category similarity. Suppose a
regular user asks -\textit{ a 1997 Toyota Camry V6 needs what size tires?} In
this query, identifying \textit{Camry V6} as a subclass of \textit{Toyota} cars
and a predecessor of \textit{automobile models} is important, because this will
help us grouping stored queries.


This paper discusses related question answering and ontology generation systems in section \ref{sec:relatedwork}. Section \ref{sec:systemarch} explains the Morpheus system and its implementation. Section \ref{sec:results}, we describe current results and their
interpretation. We conclude by describing the global impacts
of our approaches in QA systems.

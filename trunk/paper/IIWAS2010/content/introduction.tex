\section{Introduction}


4. Motivation 

1. What are we proposing?  

2. Challenges

  2.1 User query processing 

  2.1 Need of an ontology 

3. Realm based question answering 

3. About the paper structure  



When traveling though a jungle to a destination, it is easy to get
lost.  The first person to go somewhere may make a number of mistakes
in trying to find the best path to their destination.  However, those
who come later will find it easy to reach that destination if a
well-marked trail has been created. Olsen and Malizia describe this
idea as \emph{exploiting trails} \cite{5379671}.  Rather than treating
a user's search as a unique entity and working without an example
context, one can exploit the fact that a similar search may have been
performed already.  In one study, almost 40 percent of all queries
were found to be repetitions of previous queries\cite{1277770}. This
indicates the reuse of prior searches is a way to optimize the search
process.  We have built a prototype system for answering questions
using deep-web sources that employs previously answered questions to
obtain new answers.  Using our analogy, the Morpheus question
answering system follows path-finders to their destinations and not
only marks the trail, but also provides a taxi service to take
followers to similar destinations.

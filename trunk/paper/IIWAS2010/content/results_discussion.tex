\section{Results}
\label{sec:results}

% 3-4 queries 
% Human made ontology 
% Topic Models 

As an initial step we created a man-made realm-ontology for the automotive realm  exploiting the Wikipedia pages, categories, and WordNet synsets. For each of the categories in the realm-ontolgy we built corpora by grabbing the corresponding Wikipedia pages. The following section describes question answering process for a user query ``a 1997 Toyota Camry V6 needs what size tires?''. 

The Morpheus NLP engine parses this query into the constructs: WH:what, Descriptive Information:1997 Toyota Camry V6, Asking For:size tires. From the descriptive info, we generate the N-grams or terms 1997, 1997 Toyota, 1997 Toyota Camry,Toyota,Toyota Camry,Toyota Camry V6, Camry, Camry V6, and V6. For each of the terms, we determined relevant categories (non-increasing order of relevance) from the ontology corpora. The table \ref{tbl:term_categories} shows top categories and their probabilities for each of the query term: 

\begin{table}[h]\footnotesize

\begin{tabular}{l | l | r}
Term & Category & $P(Category|Term)$ \\
\hline
1997 & Sedans & 4.04e-09\\
1997 Toyota & Engines & 7.90e-14\\
Toyota  & Sedans & 3.48e-08\\
Toyota Camry & Sedans & 1.20e-10\\
Toyota Camry V6 & Coupes & 1.38e-13\\
Camry & Sedans & 3.12e-09\\
Camry V6 & Coupes & 1.38e-13\\
V6 & Sedans & 4.46e-08\\
\hline
\end{tabular}        

\caption{Terms' top categories and probabilities}
\label{tbl:term_categories}   


\end{table}

\subsection{Discussion}


% 2. Current stage of the project 
% 3. Expected contribution to the topics of interest supported by iiWAS2010 

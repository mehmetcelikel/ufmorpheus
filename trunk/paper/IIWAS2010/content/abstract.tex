\begin{abstract}

The Morpheus question answering system uses deep web search to answer user queries. Queries are represented by a semi-structured format that contains query terms and referenced classes within a specific ontology. Morpheus answers questions by using methods stored in prior successful searches answering similar queries. We rank stored methods based on a similarity measure defined on assigned classes of queries. This measure depends on class heterarchy in a realm-based ontology and the associated text corpora. Finally, we revisit the prior search pathways of the stored searches to construct possible answers. The realm-based ontologies are created using the Wikipedia pages, associated categories, and the synset heterarchy of WordNet. This paper describes the entire process with emphasis on the matching of user queries to stored answering methods.


\end{abstract}

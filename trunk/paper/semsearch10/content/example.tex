%jnw

The end-user employs the Morpheus question-answering system much like a search engine with a natural-language interface. One can type a question in a text box and receive a response from the system. The system uses previously answered queries to answer new questions.  The reused questions have a similar form and the same ontological realm, and employ sources that will yield relevant solutions.  We focus our reuse on the sources that are hidden behind web forms, or the \emph{deep web}.

Suppose the user poses the question ``A 1997 Toyota Camry V6 needs what size tires?'' The answer (205-65-15) will be presented, together with information describing assumptions about the question, the likely relevance of the answer provided,
the source of information, and the method of answer production.

Each answer to a question is created using a \emph{query response
  method} (QRM). A QRM is constructed when a special user, known as a
\emph{guide}, answers a question using an instrumented web browser (the
\emph{query response recorder} or QRR) that records web
interactions. The system extracts the relevant artifacts of the
guide's actions and stores them in the QRM. The QRM is characterized
by a \emph{semi-structured query} (SSQ), a schematic representation of
the information contained in the natural language question, namely the input and output
contexts---the \emph{who, what, when,} and \emph{where}
categories---that characterize the question.  In addition, the QRR has categorized a generalized representation of the guide's browser
interactions.  The \emph{query response evaluator} (QRE) evaluates a
query by binding specific arguments to its input parameters and
replaying the generalized web interactions the guide used to answer
the original question.

To answer a user's question, we must identify likely contextual
categories associated with the terms in the question to form an SSQ,
find those QRMs having contextually compatible SSQs, and then use the QRE to
generate a response.


In what follows, we detail the elements of the system.  We first
discuss how the processing of natural language queries generates SSQs.
Next, we discuss the recording of guide behavior with the QRR.  We
then discuss how the categorical heterarchies associated with
ontological realms are constructed in a semi-automated fashion.  We
describe the method we use to assess how well a stored QRM matches a
user SSQ.  We then discuss the QRM evaluation process and the
system's behavior for example queries. We conclude with a discussion of
future work.
 

\section{Query Resolution Recorder (QRR)}
\label{sec:qrr}
% Chris
Answering a question using the deep web requires one to navigate
through a sequence of one or more web pages. Many of the accesses involve clicks through web forms or links to resulting pages.  We have developed a model that represents the various actions a user may carry out during this process.

Web interactions can be broken down into three types:
\begin{enumerate}
\item The user highlights text that is needed in the query or forms part of the answer.
\item The user clicks a link.
\item The user fills in and submits a web form. 
\end{enumerate}

When a user submits a web form they are prompted for the classes of each of the form inputs. Here, class refers to categorization of data, e.g. name, date, etc. These classes allow the QRR to map the user's input SSQ tokens (which are also categorized) to their usage in a given form. For a clicked link, the QRR simply records the target url and nothing more. Finally, during the answer retrieval process when a user discovers all or part of the desired information on a web page, she highlights the text in question. This highlight allows the QRR to record the source of the query's answer. All information gathered during the recording process is saved in our data store. 

Any query answering process consists of a sequence of these transactions.
Additionally, data collected for inputs and outputs may be found
during any stage of the complete query process.  An output
collected during one stage of an extraction may be the input to a
subsequent stage in the extraction. The QRR builds the completed user
collection process into a Query Resolution Method (QRM).  The following is a mathematical formulation of information collected for user web interactions.

A QRM is a 5-tuple $Z = \left< \Upsilon, \Omega, P, M, R, A \right>$ such that:

\begin{enumerate}

\item $\Upsilon$ is the sequence of input categories $\left<
  \Upsilon_{1}, \Upsilon_{2}, \Upsilon_{3}, \Upsilon_{4} \right>$
  where $\Upsilon_{1}$ represents the \emph{who} context of the
  associated query, $\Upsilon_{2}$ represents the \emph{what} context,
  $\Upsilon_{3}$ represents the \emph{when} context, and
  $\Upsilon_{4}$ represents the \emph{where} context.

\item $\Omega$ is the sequence of output categories $\left<\Omega_{1},
  \Omega_{2}, \Omega_{3}, \Omega_{4}\right>$ where $\Omega_{1}$
  represents the \emph{who} context of the associated query,
  $\Omega_{2}$ represents the \emph{what} context, $\Omega_{3}$
  represents the \emph{when} context, and $\Omega_{4}$ represents the
  \emph{where} context.

\item $P$ is an ordered list of web pages $P = \left<P_1,P_2,...,
  P_n\right>$ where $P_h =
  \left<U_h,\left<I_{h_1},I_{h_2},...,I_{h_u}\right>,\left<O_{h_1},O_{h_2},...O_{h_v}\right>\right>$
  for $1 \leq h \leq n$, $u = \left| I_h \right|$, $v = \left| O_h
  \right|$. $P_h$ is a triple comprised of a URL $U_h$, a sequence of
  input arguments, and a sequence of output results.  \\ Let $I_{\$} =
  \bigcup_{j=1}^{4} I_{\downarrow_j}$ be the set of the selector
  expressions for the inputs of a query and $O_{\$} =
  \bigcup_{j=1}^{4} O_{\downarrow_j}$ be the set of selectors
  expressions for the outputs of a query.  \\ Let $K$ be the set of
  all string constants.
\begin{itemize}
\item URI $U_1 \in K$ is a string and \\ for $1 < g \leq n$, URI $U_g
  \in K \cup \left( \bigcup^{g-1}_{p=1} \left(
  \bigcup^{\left|O_p\right|}_{h=1}O_{p_h} \right)\right)$ where $1
  \leq h \leq \left| O_p \right| $,

\item for $1 < h < \left| I_1 \right|$, input $I_{1_h} \in K \cup
  I_{\$} \cup O_{\$}$ and \\ for $1 < g <= n$, then for $1 < h <
  \left| I_g \right|$, input $I_{g_h} \in K \cup I_{\$} \cup O_{\$}
  \cup \left( \bigcup^{g-1}_{p=1} \left(
  \bigcup^{\left|O_p\right|}_{h=1}O_{p_h} \right)\right)$.
\end{itemize}

\item $M$ is a map between $\Upsilon$, $K$, $\Omega$ to page list inputs $I_h$.

\item $R$ is an ontological realm. 

\item $A$ is the sequence of outputs from $\Upsilon$, $K$, and
  $\Omega$ representing the answer.

\end{enumerate}

%-------------------------------------------------------------------------------

%\subsection{Generalization}
% Christan
The response pages created by querying web forms are dynamically
generated with a templated structure. This generalization per page is
done using the GenPath algorithm developed by Badica et
al. \cite{Badica06}. Additionally, the community may be able to answer
a particular SSQ using many different page interactions. Therefore, we
use a method of generalizing extraction path to a result across
pages. This method removes unnecessary page interactions and finds the
shortest path to the result pages.

Websites wrapped by a system such as this are highly susceptible to
structural changes \cite{TanZMG07}. Generalizing the extraction paths
helps overcome small structural changes. If a website changes to the
point where a QRM is no longer useful, the QRM is not considered
fresh. User ratings of QRM freshness can be employed in the SSQ matching
process detailed in Section \ref{sec:ssq}.

%-------------------------------------------------------------------------------

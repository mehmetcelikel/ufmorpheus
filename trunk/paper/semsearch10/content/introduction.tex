\section{Introduction}
% Christan
When traveling though a jungle to a destination, it is easy to get lost.  If you are the first person to take a path then you have to make a lot of mitakes in order to find the best path to a destination.  However, if you are the \emph{n}th person to take a path, you may be able to follow the path that was left by the \emph{n-1} other people to walk the path.  An article by Olsen and Malizia describe this idea as \emph{exploiting trails} \cite{5379671}.  We use this insight to to improve the efficency of retrieval in semantic search.

Instead of treating each user search as a unique search, it is beneficial to assume that a similar search has been previously performed.  Researchers have shown that almost 40 percent of the all queries are repeats \cite{1277770}.  This means there is a large open area for optimization for search engines.  We have built a prototype system for deep web question answering that uses previously answered questions to obtain new answers.

In this paper we describe a system for recording the answer to user queries and executing solutions to similar queries.  To support this we use a structured query representation.  The structured representation allows queries to be parameterized so that similar queries may be discovered at query-time.  To recognizing these queries we build a semantic heterarchy and define algorithms for matching the classes.

The paper proceeds as follows:
\begin{itemize}
\item Discussion of the query structure.  We call this structure an \emph{semi-structured query} SSQ.
\item Next we discuss how we create the class heterarchy to support the semantic comparisons.
\item In section \ref{sec:Building Query Store} we discuss how we record user web queries and correctly annotate them.
\item Next, we discribe how we match rank 
\item ....
\item We will conclude by discussing the examples of the system.
\end{itemize}

\subsection{Category Heterarchy} 
% Author: Clint 
Since, our representation of semi structured query [citation needed]
is in OWL ontology format, and our aim is to compare and rank the
queries in this format, we had to define a similarity measure between
ontologies. The defined ontology class comparison measure "class
divergence" [citation needed] assumes that we have an ontology store
with unambiguous interpretation of classes and strict hierarchical
subclass super class relationships and the classes to be compared
should belong to it. We cannot find any robust existing ontology
library of that nature for this purpose. So we decided to build a new
ontology store and as an initial step the categories from DBPedia
store [citation needed] are used. Even though, some of these
categories are loosely coupled and not fully agreed upon the sub
category and super category relationships, this DBPedia has enormous
amount of data in linked form [citation needed]. The OWL ontology
[citation needed] format is followed in this process. The DBPedia
categories are converted into OWL ontology classes and the super class
- sub class relationships are built using the OWL super class sub
class axioms [citation needed]. Another reason for selecting OWL for
this purpose is that we can make use of the existing APIs for building
and reasoning [citation needed] the ontology relationships.


%-------------------------------------------------------------------------------



\subsection{Associating Terms with Categories} %joir-dan The
subcategories of the category Automobiles (… Driving, Traffic Law,
Automobiles by Decade, Auto Racing…) can be seen as grabbing
information specific to the category. The supercategories (Road
Vehicles) place the category in a higher perspective, abstractly
describing basketball as a concept. The spousal categories
(...Motorsport, Individual Sport...) are an abstract description of
the subcategories.


Clustering the categories alone does not guarantee adequate corpus
creation. Since the above model relies on the subcategories to
describe specifics of a given category, the ideal situation would be
that the subcategories have many more pages than the supercategories
and that there is a significant amount of information per page.  In
the case of Wikipedia, there are an average of 25.7 pages per category
\cite{1321474}. The assumption for collecting the training data is the
more subcategory data we have, the more detailed our description of
our category. Considering that there are many categories that have
much more than 26 pages, we instituted a PlusOne mechanism to extract
extra information. This mechanism checks if a subcategory has an
adequate number of pages α to fully describe its subtopic. We set this
number to ten times the average, or 260. $Level$ represents how many
extra levels we are to go down in the hierarchy, if
necessary. SubjectSet is the overall set of all pages associated with
the blanket structure along with any plusOne subjects obtained.

\begin{lstlisting}[language=Python,frame=none,tabsize=2,caption=BlanketSubjectRetreival,label=BlanketSubjectRetreival,basicstyle=\small]
def BlanketSubjectRetreival(blanket, maxLevel):
	for category in blanket:
		subjectSet.add(grabsubjects(category))
		diff = afterSize - beforeSize
		if diff <= a and category is subcategory:
			PlusOneMechanism(category, maxLevel)
	return subjectSet
\end{lstlisting}

\begin{lstlisting}[language=Python,frame=none,tabsize=2,caption=PlusOneMechanism, label=PlusOneMechanism,basicstyle=\small]
def PlusOneMechanism(category, level):
	plusOneSet = grabPlusOneSubcategories(category)
	for subcategory in plusOneSet:
		subject.add(grabSubjects(subcategory))
		diff = afterSize - beforeSize
		if diff <= a and level > 0:
			PlusOneMechanism(subcategory, level-1)


\end{lstlisting}

The overall mechanism does not take into account the amount of
information within each page, but generally that can be ignored when
considering the size of the blanket corpus when we extract the text
from the Wikipedia subject pages. Using this method on the category
Automobiles, we can construct a corpus of 2255 subject pages and
6,084,260 terms.

When a category's corpus is built, we then build our n-gram frequency
distributions and store them in a database for the NLP processing. For
the purpose of calculating term importance to a blanket, we aggregate
the frequencies of all blankets and store this information as
well. For this system, term importance refers to the prior probability
of a category given an n-gram. We calculate this using Bayes Rule,
since we can easily obtain the n-gram-category and n-gram-vocabulary
probabilities. The higher the probability, the more likely a term is
referring to a category.

\begin{equation}
P (category | term) = \frac{P(term | category) P(category)}{P(term)}
\end{equation}

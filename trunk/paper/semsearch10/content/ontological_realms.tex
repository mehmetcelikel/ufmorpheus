\section{Representing and Ontological Realms}
Two types of ontologies are used by our system:
\begin{enumerate}
\item Realm ontologies, comprising a heterarchy of categories that
  characterize terms appearing in queries.
\item SSQ ontologies, describing the form of particular
  realm-specific queries.
\end{enumerate}
For any particular ontological realm, the number of potential
terms and their associated categories can be quite large.
Construction of a realm ontology and the terms associated
with its categories would be an extremely laborious task
for any non-trivial realm. For example, the realm of automobiles (which
we use for illustration in this paper) contains 
over XXXX categories and XXXX terms. We were unable to find
an appropriate ontology for this realm, so we constructed it
using categories from the DBPedia \cite{Auer07dbpedia:a}.
The primary benefit of using the DBPedia for this purpose is
that not only does it supply a categorical heterarchy, but it
also links to Wikipedia documents which provides us with a large
realm-specific document corpus.

\subsection{Category Heterarchy} 

To support our process, we map DBPedia categories into classes in our realm ontology with the \textit{superclass} and \textit{subclass} relationships built using the DBPedia properties \textit{broader} and \textit{narrower}. By using OWL, we can exploit the OWL API for matching SSQ ontologies that reference these realm ontologies.

%Start with a collection of relevant categories 

Once the semantic hierarchy is created, the next step is to create
training data for the system. The task is then to grab the pages,
build a corpus relating to a given category, and then compile the
information in a meaningful way. The overall goal is to grab enough
information to properly characterize a category while limiting the
amount of unnecessary content within the corpus. The problem with this
approach is the page density varies greatly across the
heterarchy. Each category within the DBpedia/Wikipedia heterarchy
contains a set of information pages, P, corresponding to the concepts shared by the category. Since the categorical structure is a directed acyclic graph \cite{Suchanek07yago:a}, we use of a Markov Blanket \cite{Friedman97bayesiannetwork} to describe
a particular category. Our assumption is that a
node can be fully described by its ancestor, descendant, and spousal
nodes (parent node of subcategory not equal to the category). While
this does not completely solve the issue of a category residing in a
sparse area of the graph, the amount of information that can be
obtained per blanket is usually more helpful than per category.


The subcategories of the category Automobiles (Driving, Traffic Law,
Automobiles by Decade, and Auto Racing) can be seen as grabbing
information specific to the category. The super categories (Road
Vehicles) place the category in a higher perspective, abstractly
describing basketball as a concept. The spousal categories
(...Motor Sport, Individual Sport...) are an abstract description of
the subcategories.


Clustering the categories alone does not guarantee adequate corpus
creation. Since the above model relies on the subcategories to
describe specifics of a given category, the ideal situation would be
that the subcategories have many more pages than the super categories
and that there is a significant amount of information per page.  In
the case of Wikipedia, there are an average of 25.7 pages per category
\cite{1321474}. The assumption for collecting the training data is the
more subcategory data we have, the more detailed our description of
our category. Considering that there are many categories that have
much more than 26 pages, we instituted a PlusOne mechanism to extract
extra information. This mechanism checks if a subcategory has an
adequate number of pages α to fully describe its subtopic. We set this
number to ten times the average, or 260. $Level$ represents how many
extra levels we are to go down in the hierarchy, if
necessary. SubjectSet is the overall set of all pages associated with
the blanket structure along with any plusOne subjects obtained.

\begin{lstlisting}[language=Python, frame=none, tabsize=2, caption=BlanketSubjectRetreival, label=BlanketSubjectRetreival, basicstyle=\small]
def BlanketSubjectRetreival(blanket, maxLevel):
	for category in blanket:
		subjectSet.add(grabsubjects(category))
		diff = afterSize - beforeSize
		if diff <= a and category is subcategory:
			PlusOneMechanism(category, maxLevel)
	return subjectSet
\end{lstlisting}

\begin{lstlisting}[language=Python, frame=none, tabsize=2, caption=PlusOneMechanism, label=PlusOneMechanism, basicstyle=\small]
def PlusOneMechanism(category, level):
	plusOneSet = grabPlusOneSubcategories(category)
	for subcategory in plusOneSet:
		subject.add(grabSubjects(subcategory))
		diff = afterSize - beforeSize
		if diff <= a and level > 0:
			PlusOneMechanism(subcategory, level-1)


\end{lstlisting}


\subsection{Associating Terms with Categories}
%joir-dan

The overall mechanism does not take into account the amount of
information within each page, but generally that can be ignored when
considering the size of the blanket corpus when we extract the text
from the Wikipedia subject pages. Using this method on the category
Automobiles, we can construct a corpus of 2255 subject pages and
6,084,260 terms.

When a category's corpus is built, we then build our n-gram frequency
distributions and store them in a database for the NLP processing. For
the purpose of calculating term importance to a blanket, we aggregate
the frequencies of all blankets and store this information as
well. For this system, term importance refers to the prior probability
of a category given an n-gram. We calculate this using Bayes Rule,
since we can easily obtain the n-gram-category and n-gram-vocabulary
probabilities. The higher the probability, the more likely a term is
referring to a category.

\begin{equation}
P (category | term) = \frac{P(term | category) P(category)}{P(term)}
\end{equation}

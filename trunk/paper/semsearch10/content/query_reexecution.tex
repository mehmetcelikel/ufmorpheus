\subsection{Query Re-execution}
% Chris
	I ASSUME CLINT HAS COVERED QRM SELECTION ALREADY
	When a user submits a new query to the Morpheus system, one or more stored 
Query Resolution Methods are executed to obtain results. These QRMs are sorted based
on their relevance to the query at hand and are chosen by the ontology matching subsystem.
	A QRM consists of an XML script which instructs the Query Resolution Executor how to 
obtain the required result. The QRE may "click" links, submit forms or highlight and save
important pieces of html, including answers. Once it has visited all pages included in the 
script the QRE returns the HTML snippet containing the answer. This answer can then be
rendered in a browser. As mentioned above, normally a list or relevant QRMs will be given
to the QRE which will in turn be executed using the provided SSQ, the results will be 
displayed in order for the user. 
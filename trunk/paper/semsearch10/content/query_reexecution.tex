\subsection{Query Re-execution}
% Chris
When a user submits a new query to the Morpheus system, one or more
stored Query Resolution Methods are executed to obtain results. A QRM
contains an XML script that represents an algorithm that will obtain
the required result.  The evaluation of this script by the QRE
simulates the behavior of a human browsing the web, that is, clicking
buttons to follow links, submitting forms, highlighting and cutting
and pasting text, and constructing an answer in the form of a text
string.

At present, the GET form submission method is used.  There is greater
complexity in dealing with web forms than with links. For a given
form, there may be text fields, drop-down boxes, radio buttons etc.
To support form submission in such circumstances the QRE takes
different actions based on the input type. For text fields, simply
adding a key/value pair to the querystring suffices. For a drop-down
box, the QRE extracts the form element find the appropriate value to
include in the querystring.

All input data as well as data highlighted during the evaluation are
kept in an internal table. The QRM script contains keys that allow the
QRE to map SSQ inputs from the table to the appropriate form
inputs. Furthermore, highlighted text segments collected during
execution of the script are stored in the table.  Once all actions in
the script have been carried out, the QRE returns textual answer this
process has created. This answer can then be rendered in a
browser.

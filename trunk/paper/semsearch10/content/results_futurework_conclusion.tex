\section{Results}
% Walk through of example query

\section{Future Work}

The largest limitation to our current work is the the QRR.  In order to efficiently answer new user questions the systems needs a large amount of user annotated queries.  However, a significant amount user interaction is required to understand the answer retrieval process.  By continuing to studying the methods of user data collection we plan to develop a model to support a unsupervised question resolution process.  This will increase the scalability and efficiency of the system.

Because each QRM may be treated as a function, we plan on leveraging previous work to allow new questions to be solved by connecting together several QRMs \cite{morpheus1, transformscout}.  This method would expand the types of queries that are answerable be the system.

Currently, the category hierarchy that is generated from the DBpedia categories and our SSQ ontologies are stored in separate XML files. We are in the process of storing these ontologies into an efficient ontology database. Similarly, employing user input brings about the likelihood that redundant SSQ ontological elements will be inserted into our ontology store. We are working on merging the similar SSQs [\label{sec:ssq}] based on the SSQ realms and other ontological structural information so that we can remove the redundancy in the database.

In addition, we cannot fully depend on the DBpedia categories, since the relevant data available in DBpedia store is insufficient to represent the real world queries belongs to all the realms. [e.g. musical bands]. We are planning to define new categories from the deep web [or find out new sources to fill those \textit{weak} realms] in our ontology hierarchy.      

Researchers have stated 39, 57, and 63 percent of web queries are unique \cite{1277770,331405,621942}.  We believe that an even higher number of those are parametrizable, meaning, a set of queries are all related and may be answered using a single QRM.  It is difficult to discover this property.  Previous work suggest techniques to find related queries but it will take a new method to discover parametrizable queries.

% Sort classes based on the prior 

\section{Conclusion}

In this paper we provide a discussion on a new type of question answering system that uses previously answered questions user.  We described our method of modelling a users question answering process and how we are able to discover semantically similar queries.

MORE

\section{Acknowledgements}

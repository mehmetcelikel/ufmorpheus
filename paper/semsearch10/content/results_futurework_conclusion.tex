\section{Results}
% Walk through of example query

\section{Future Work}

Currently, the category hierarchy that is generated from the DBpedia categories and our SSQ ontologies are stored in separate XML files. We are in the process of storing these ontologies into an efficient ontology database. Similarly, employing user input brings about the likelihood that redundant SSQ ontological elements will be inserted into our ontology store. We are working on merging the similar SSQs [\label{sec:ssq}] based on the SSQ realms and other ontological structural information so that we can remove the redundancy in the database.

In addition, we cannot fully depend on the DBpedia categories, since the relevant data available in DBpedia store is insufficient to represent the real world queries belongs to all the realms. [e.g. musical bands]. We are planning to define new categories from the deep web [or find out new sources to fill those \textit{weak} realms] in our ontology hierarchy.      

- Researchers have stated 39, 57, and 63 percent of web queries are unique \cite{1277770,331405,621942}.  We believe that an even higher number of those are parameterizable, meaning, a set of queries are all related and may be answered using a single QRM.  We plan to investigate this.

% Sort classes based on the prior 

\section{Conclusion}

\section{Acknowledgements}

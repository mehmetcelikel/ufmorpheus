\begin{abstract}
The Morpheus question answering system uses deep web search to answer
user queries. Queries are represented as OWL ontologies that reference
categories within specific ontological realms.
The system answers questions by using methods identified in prior
successful searches answering similar queries.
Stored methods are ranked based on their ontological similarity to the
question being answered.
The similarity measure depends on finding the distance between
ontological categories associated with their search terms.
The DBpedia and Wikipedia provide information we use to identify the likely
categories associated with those terms.
Finally, we replay appropriately re-parameterized relevant prior searches
to construct possible answers.
The paper describes this entire process with emphasis on the matching of
user queries to stored answering methods.
\end{abstract}

\section{Query Re-execution}
% Chris
When a user submits a new query to the Morpheus system, one or more
stored QRMs are executed to obtain results. A QRM contains an XML script representing an algorithm that will obtain the required result.  The evaluation of this script by the QRE simulates the behavior of a human browsing the web, that is, clicking buttons to follow links, submitting forms, highlighting data, cutting and pasting text, and constructing an answer in the form of a text string.

At present, the GET form submission method is used.  There is greater
complexity in dealing with web forms than with links. For a given
form, there may be text fields, drop-down boxes, and among other components.
In order to support form submission in such circumstances, the QRE performs
different actions based upon the input type. For text fields, simply
adding a key/value pair to the query string suffices. For a drop-down
box, the QRE extracts the form element to find the appropriate value to
include in the query string.

All input data are kept in an internal table. The QRM script contains keys that allow the QRE to map SSQ inputs from the table to the appropriate form
inputs. Furthermore, highlighted text segments collected during
execution of the script are stored in the table.  Once all actions in
the script have been carried out, the QRE returns the textual answer this
process has created. This answer is then be rendered in a browser displayed to the user.

\section{Natural Language Queries}
% Joir-dan stuff
The first step in answering a user query is to determine what, in
fact, is being asked.  Any system that supports user queries places
some constraints on the required input.  These constraints can be
relatively minimal ,``Users must type their input on a 104-key
keyboard in English,'' or highly restrictive, ``Users must constuct
fully categorized semi-structured queries employing only those terms and
categories supplied by our database.''  Our initial goal in Morpheus
is two-fold:

\begin{enumerate}
\item Implement an initial natural language interface that allows
  users to make queries in a loosely constrained English language
  subset.
\item Provide a framework that can support a robust
  interface that employs statistical language processing
  methods.  
\end{enumerate}

At present, we can process English language questions posed in
\emph{answer form}, for example, ``The closest Ford dealer to the University
of Florida is where?'' This requires minimal structural transformation in order
to represent the semantic information contained within the question.  We
translate these questions into semi-structured queries that associate
English language terms and modifier phrases with contexts.

Input contexts of such a semi-structured query are generally given as
terms (particular dates, names of cars, etc.) while output contexts
are given as descriptions of categories or contexts, i.e.,
\emph{tire type, what date, where}. Thus the inputs of the
user SSQ are uncategorized, while their outputs are categorized.  The
SSQs associated with a QRM are fully categorized (as will be discussed
in section \ref{sec:qrr}). To match a user SSQ to existing QRMs, we
must be able to match their SSQ input and output contexts.  This
requires matching the input terms to their categories, a process we
discuss in section \ref{sec:terms}.

\section{Introduction}

% Outline 
% ---------------------

% 1. Motivation 
% 2. What are we proposing?  
% 3. Challenges
% 		3.1 User query processing 
% 		3.1 Need of an ontology 
% 4. Realm based question answering 
% 5. About the paper structure 

When traveling though a jungle to a destination, it is easier to get
lost.  The first person to go somewhere may make a number of mistakes
in trying to find the best path to their destination.  However, those
who come later will find it easy to reach that destination if a
well-marked trail has been created. Olsen and Malizia describe this
idea as \emph{exploiting trails} \cite{5379671}.  Rather than treating
a user's discovery experience as a unique entity and working without an example
context, one can exploit the fact that a similar search may have already been
performed.  In one study, almost 40 percent of all queries
were found to be repetitions of previous queries\cite{1277770}. This
indicates the reuse of prior searches is a way to optimize the search
process.  Morpheus is a question answering system motivated by reuse of prior
web search pathways to yield an answer to a user query. Using our analogy,
Morpheus follows path-finders to their destinations and not only marks the
trail, but also provides a taxi service to take followers to similar
destinations.

The Morpheus system employs two distinct roles. One type is called a
\textit{guide} who types queries in the Morpheus web interface and
searches for an answer to the query in an instrumented web browser that can
track answer pathways for a given query. This web tracking tool stores the query
and necessary information to revisit the pathways to the page where the guide 
found the answer. It also provides a mechanism for guides to highlight the answers.
The other role is a \textit{user}, who searches with the Morpheus system much
like a regular search engine, having a natural language interface. The users type a
question in a text box and receive a response from the system. The system
exploits previously answered queries to these answer new questions.

To exploit the prior searches one should represent queries in an elegant way so
that similar queries can be found efficiently and their stored search path
revisited to produce the answer to the user. Morpheus represents user-query details
in a semi structured format called a Semi-Structured Query or \textit{SSQ}. It
includes a user query in natural-language format, relevant terms for the query,
place holders for the term \textit{categories}. A \textit{category} represents a
topic or class to which a term belongs. For example, the term \textit{tire} can be
classified into the category \textit{Automotive Parts}.

Morpheus assumes the term categories are classes of a
\textit{consistent} realm-based ontology. By consistant we mean the ontology has two
properties: (i) singly-rooted heterarchy (acyclic digraph with unique 
ancestor of all nodes) (ii) the subclass/superclass relations have meaningful 
semantic interpretations. This is important because, our proposed similarity 
measure for categories is based on the heterarchically
structured category ontologies. Moreover, when a regular user enters a query,
Morpheus ranks SSQs in the store based on this catgeory similarity. Suppose a
regular user asks -\textit{ a 1997 Toyota Camry V6 needs what size tires?} In
this query, identifying \textit{Camry V6} as a subclass of \textit{Toyota} cars
and a predecessor of \textit{automobile models} is important, because this will
help us grouping stored queries. Similarly, a clean ontology-taxonomy can
improve the recall of such queries by supporting transitivity and other types of
inference like faceted browsing \cite{Wu2008}.   

[[Need a paragraph for wiki page categorization and ontology construction]]
In this paper, we propose an automatic web page categorization or 
classification and recommendation using a machine learning technique, topic modeling.
In addition, we also describes how we can use the categorized pages for realm-ontology 
creation.   

This paper starts by discussing state-of-the-art systems in the area of
question answering and ontology generation. The system
architecture section explains the Morpheus system and its implementations
details. In the results section, we describe current results and its
interpretations. Finally, we conclude the paper by describing the global impacts
of our approaches in QA systems, WWW page categorization, and ontology
generators.

\begin{abstract}

When users search the deep web, the essence of their search is often found in a previously answered query. The Morpheus question answering system reuses prior searches to answer similar user queries. Queries are represented in a semi-structured format that contains query terms and referenced classes within a specific ontology.  Morpheus answers questions by using methods from prior successful searches.  The system ranks stored methods based on a similarity quasi-metric defined on assigned classes of queries. Similarity depends on the class heterarchy in an ontology and its associated text corpora. Morpheus revisits the prior search pathways of the stored searches to construct possible answers. The realm-based ontologies are created using Wikipedia pages, associated categories, and the synset heterarchy of WordNet. This paper describes the entire process with emphasis on the matching of user queries to stored answering methods.


\end{abstract}

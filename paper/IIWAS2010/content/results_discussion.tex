\section{Results}
\label{sec:results}

% 3-4 queries 
% Human made ontology 
% Topic Models 

As an initial step we created a man-made realm-ontology for the automotive realm
 exploiting the Wikipedia pages, categories, and WordNet synsets. For each of
the categories in the realm-ontolgy we built corpora by grabbing the
corresponding Wikipedia pages. The following section describes question
answering process for a user query ``a 1997 Toyota Camry V6 needs what size
tires?''. 

The Morpheus NLP engine parses this query into the constructs: WH:what,
Descriptive Information:1997 Toyota Camry V6, Asking For:size tires. From the
descriptive info, we generate the N-grams or terms 1997, 1997 Toyota, 1997
Toyota Camry,Toyota,Toyota Camry,Toyota Camry V6, Camry, Camry V6, and V6. For
each of the terms, we determined relevant categories (non-increasing order of
relevance) from the ontology corpora. The table \ref{tbl:term_categories} shows
top categories and their probabilities for each of the query term: 

\begin{table}[h]\footnotesize

\begin{tabular}{l | l | r}
Term & Category & $P(Category|Term)$ \\
\hline
1997 & Sedans & 4.04e-09\\
1997 Toyota & Engines & 7.90e-14\\
Toyota  & Sedans & 3.48e-08\\
Toyota Camry & Sedans & 1.20e-10\\
Toyota Camry V6 & Coupes & 1.38e-13\\
Camry & Sedans & 3.12e-09\\
Camry V6 & Coupes & 1.38e-13\\
V6 & Sedans & 4.46e-08\\
\hline
\end{tabular}        

\caption{Terms' top categories and probabilities}
\label{tbl:term_categories}   

\end{table}

We used this top category of each of the terms in the candidate SSQ and
determined the category divergence with the qualified SSQ categories in the QRM
store. We assumed that all of them belong to the same realm or domain (but, it
is different in real time). In the end, we combine the divergence measure and
ranked QRMs based on the relevance score. The table \ref{tbl:ranked_queries}
shows the top ranked queries that belong to the QRMs in store for the above
query.

\begin{table}[h]\footnotesize

\begin{tabular}{l | r}
Query & Score\\
\hline
The tire size of a 1997 Toyota Camry V6 is what? & 0.11\\ 
\hline
\end{tabular}        

\caption{Top ranked queries and their relevance score}
\label{tbl:ranked_queries}   

\end{table}


%\subsection{Discussion}
\section{Conclusion}
This query matching strategy assumes that query terms are available in the
category corpus. The candidate query's category annotation will fail if the
query terms are not available in the corpus. We are working on improving this
approach by using a probabilistic approach as used in the Topic
Models\cite{Blei2003latentdirichlet}. Similarly, the Morpheus ontology building
process needs categorized web pages. As explained in section
\ref{sec:ontology_generators}, the Wikipedia page categories have some
limitations for this purpose. Without a super vision, the page categorization is
often a challenging problem, because topics in the pages that belong to a realm
(e.g. automotive) usually overlap. So we are experimenting the applicability of
probabilistic topic models into this web page categorization, and automatic
generation of a realm-ontology based on the topics being extracted.            
 


% 2. Current stage of the project 
% 3. Expected contribution to the topics of interest supported by iiWAS2010 

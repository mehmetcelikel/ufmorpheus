\section{Related Work}
\label{sec:relatedwork}

\subsection{Question Answering Systems} 

The earliest question answering systems such as BASEBALL \cite{Green1961} and Lunar \cite{woods1973} had \emph{closed domains} and \emph{closed corpora}, that is, they supported finitely many questions on copora containing a fixed set of documents. Morpheus uses the web as its dynamic, open corpus and examines deep web sources to answer questions.

Several community QA systems have been developed, such as Yahoo! Answers \cite{yahooanswers2008}. These sites enlist a community of users to answer questions posed by other users.  This idea has led to socially integrated QA systems such as Aardvark \cite{vark2010}. In addition to having questions answered by users, Aardvark can contact potential question answerers through media beyond web pages (e.g. instant messages, mobile phones).  This allows for faster responses to questions.  Morpheus relies on its user interface and referencing previously answered questions.

\subsection{Ontology generators} 
\label{sec:ontology_generators}

% This section needs to be reworked. It should be about 1 paragraph. It should define
% what these (ontology, dbpedia, wordnet) mean to morpheus
% What rigors boone invest to my pyramids? <<-- randomly generated


%An ontology formally models real world \textit{concepts} and their relationships. A concept or class in an ontology gives us an abstract and simplified view of the world\cite{Gruber1993} in a machine-readable and language-independent format. The relationships and attributes of an ontological concept are defined using \textit{properties} and \textit{property restrictions}. An ontology definition contains all of these: classes, properties, and restrictions. A class in an ontology can have multiple \textit{super classes} and \textit{sub-classes}. Finally, a \textit{knowledge base} represents an ontology together with instances of its classes.

%% use dbpedia instead of wikipedia here

The DBpedia\footnote{http://dbpedia.org} project is a community of contributors extracting semantic information from Wikipedia and makes this information available on the Web. Wikipedia semantics includes info-box templates, categorization information,  images, geo-coordinates, links to external web pages, disambiguation pages, and redirects to pages in Wiki markup form \cite{Bizer2009}.  DBpedia does not define any new relations between the Wikipedia categories.  

%Morpheus uses the DBpedia categories and its semantics to construct its ontology.
%  DBpedia represents the Wikipedia categories using skos:concepts and category relations using skos:broader \cite{Bizer2009}. In fact,  DBpedia does not define any new relations between the Wikipedia categories.  The Wikipedia categories are extracted directly from Wikipedia pages, and there  is no quality check on the resultant ontology. Similarly, additional ontological relations in DBpedia are generated from the info-box attributes and their values.

YAGO is a semi-automatically constructed ontology obtained from the Wikipedia pages, info-boxes, categories, and WordNet\footnote{http://wordnet.princeton.edu} synsets heterarchy \cite{Suchanek2009phd}. YAGO uses the Wikipedia page \emph{titles} as its ontology \emph{individuals} and \emph{categories} as its ontology \emph{classes}. YAGO uses only the nouns from WordNet and ignores the WordNet verbs and adjectives.  YAGO discovers connections between WordNet synsets and Wikipedia categories, parsing the category names and matching the parsed category components with the WordNet synsets. Each Wikipedia category not having a WordNet match is ignored in the YAGO ontology.  The ontology's heterarchy is built using the \textit{hypernym} and \textit{hyponym} relations of the WordNet synsets.  

We use YAGO's principles to construct ontologies that provide similarity measures for answering questions within the same domain.  Thus far, these ontologies can be used to classifiy terms, however their classes do not always appropriately categorize query parameters.  It is necessary to provide an appropriate level of class granularity.  Therefore, we are investigating  methods of identifying classes and their instances from deep web forms and documents.

%Morpheus uses many of the same ideas made when using YAGO to develop its own ontology as explained in Section \ref{sec:ontology_corpora}.

%However, YAGO type extraction assumes that each Wikipedia page has at least one tagged category, and the assigned Wikipedia categories are relevant to that Wikipedia page. The Wikipedia page category assignments are subjectively assigned by human editors and some Wiki pages are unassigned i.e., they have no category. In addition, we cannot fully rely on the relevance of these assigned Wikipedia page categories. In addition, YAGO concentrates mainly on the fields of people, cities, events, and movies\cite{Suchanek2009phd}.

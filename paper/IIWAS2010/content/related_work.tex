\section{Related Work}
\label{sec:relatedwork}

\subsection{Question Ansering Systems} 

\subsection{Ontology generators} 

An ontology formally models real world \textit{concepts} and their
relationships. A concept or class in an ontology gives us an abstract and
simplified view of the world\cite{Gruber1993} in a machine-readable and
language-independent format. The relationships and attributes of an
ontological concept are defined using \textit{properties} and \textit{property
restrictions}. An ontology definition contains all of
these: classes, properties, and restrictions. A class in an ontology can have
multiple \textit{super classes} and \textit{sub-classes}. Finally, a
\textit{knowledge base} represents an ontology together with 
instances of its classes.      

The DBpedia\footnote{http://dbpedia.org} project is based on extracting
semantic information from the Wikipedia and making it available on the
Web. Wikipedia semantics includes info-box templates, categorization
information, images, Geo-coordinates, links to external Web pages,
disambiguation pages, and redirects between pages in Wiki markup form
\cite{Auer07dbpedia:a, Bizer2009}. 

The extraction procedure is as follows: Wikipedia categories are represented
using skos:concepts and category relations are represented using skos:broader
\cite{Auer07dbpedia:a, Bizer2009}. In fact, DBpedia does not define any new
relations between the Wikipedia categories. The Wikipedia categories are
extracted directly from Wikipedia pages and there is no quality check on the
resultant ontology. Similarly, DBpedia uses the info-box attributes and values
as the ontological relations and its ranges.   

YAGO is a semi-automatically constructed ontology from Wikipedia and
WordNet\cite{Suchanek2009phd}. The YAGO ontology uses an automated process to
extract information from Wikipedia pages, info-boxes, and categories and to
combine this information with the WordNet\footnote{http://wordnet.princeton.edu}
synsets heterarchy. Since Wikipedia contains more individuals in the form of
Wikipedia pages than in the man-made WordNet ontology, the Wikipedia page titles
are used as the YAGO ontology individuals. YAGO concentrates mainly on the
fields of people, cities, events, and movies\cite{Suchanek2009phd}. 

In addition, YAGO uses Wikipedia categories as its ontology classes. The
ontology's heterarchy is built using the \textit{hypernym} and \textit{hyponym}
relations of the WordNet synsets. YAGO uses only the nouns from WordNet and
ignores the WordNet verbs and adjectives. The connection between a WordNet
synset and a Wikipedia category is achieved by parsing the category names and
matching the parsed category components with the WordNet
synsets\cite{Suchanek2009phd}. Those Wikipedia categories having no WordNet
match are ignored in the YAGO ontology.  
  
Moreover, YAGO type extraction is mainly based on the assumption that
each Wikipedia page has at least one tagged category, and the assigned Wikipedia
categories are relevant to that Wikipedia page. The Wikipedia page category
assignments are subjectively assigned by human editors and some Wiki pages are
unassigned i.e., they have no category. In addition, we cannot completely rely
on the relevance of these assigned Wikipedia page categories. 


\subsection{NLP Interfaces} 

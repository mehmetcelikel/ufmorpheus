% This is "sig-alternate.tex" V1.9 April 2009
% This file should be compiled with V2.4 of "sig-alternate.cls" April 2009
%
% This example file demonstrates the use of the 'sig-alternate.cls'
% V2.4 LaTeX2e document class file. It is for those submitting
% articles to ACM Conference Proceedings WHO DO NOT WISH TO
% STRICTLY ADHERE TO THE SIGS (PUBS-BOARD-ENDORSED) STYLE.
% The 'sig-alternate.cls' file will produce a similar-looking,
% albeit, 'tighter' paper resulting in, invariably, fewer pages.
%
% ----------------------------------------------------------------------------------------------------------------
% This .tex file (and associated .cls V2.4) produces:
%       1) The Permission Statement
%       2) The Conference (location) Info information
%       3) The Copyright Line with ACM data
%       4) NO page numbers
%
% as against the acm_proc_article-sp.cls file which
% DOES NOT produce 1) thru' 3) above.
%
% Using 'sig-alternate.cls' you have control, however, from within
% the source .tex file, over both the CopyrightYear
% (defaulted to 200X) and the ACM Copyright Data
% (defaulted to X-XXXXX-XX-X/XX/XX).
% e.g.
% \CopyrightYear{2007} will cause 2007 to appear in the copyright line.
% \crdata{0-12345-67-8/90/12} will cause 0-12345-67-8/90/12 to appear in the copyright line.
%
% ---------------------------------------------------------------------------------------------------------------
% This .tex source is an example which *does* use
% the .bib file (from which the .bbl file % is produced).
% REMEMBER HOWEVER: After having produced the .bbl file,
% and prior to final submission, you *NEED* to 'insert'
% your .bbl file into your source .tex file so as to provide
% ONE 'self-contained' source file.
%
% ================= IF YOU HAVE QUESTIONS =======================
% Questions regarding the SIGS styles, SIGS policies and
% procedures, Conferences etc. should be sent to
% Adrienne Griscti (griscti@acm.org)
%
% Technical questions _only_ to
% Gerald Murray (murray@hq.acm.org)
% ===============================================================
%
% For tracking purposes - this is V1.9 - April 2009

\documentclass{sig-alternate}

\begin{document}
%
% --- Author Metadata here ---
\conferenceinfo{IIWAS}{2010 Paris, France}
%\CopyrightYear{2007} % Allows default copyright year (20XX) to be over-ridden - IF NEED BE.
%\crdata{0-12345-67-8/90/01}  % Allows default copyright data (0-89791-88-6/97/05) to be over-ridden - IF NEED BE.
% --- End of Author Metadata ---

\title{{\ttlit Morpheus}: An Ontology-based Question Answering System}

%\titlenote{A full version of this paper is available as
%\textit{Author's Guide to Preparing ACM SIG Proceedings Using
%\LaTeX$2_\epsilon$\ and BibTeX} at
%\texttt{www.acm.org/eaddress.htm}}}
%
% You need the command \numberofauthors to handle the 'placement
% and alignment' of the authors beneath the title.
%
% For aesthetic reasons, we recommend 'three authors at a time'
% i.e. three 'name/affiliation blocks' be placed beneath the title.
%
% NOTE: You are NOT restricted in how many 'rows' of
% "name/affiliations" may appear. We just ask that you restrict
% the number of 'columns' to three.
%
% Because of the available 'opening page real-estate'
% we ask you to refrain from putting more than six authors
% (two rows with three columns) beneath the article title.
% More than six makes the first-page appear very cluttered indeed.
%
% Use the \alignauthor commands to handle the names
% and affiliations for an 'aesthetic maximum' of six authors.
% Add names, affiliations, addresses for
% the seventh etc. author(s) as the argument for the
% \additionalauthors command.
% These 'additional authors' will be output/set for you
% without further effort on your part as the last section in
% the body of your article BEFORE References or any Appendices.

\numberofauthors{5} 
\author{
% You can go ahead and credit any number of authors here,
% e.g. one 'row of three' or two rows (consisting of one row of three
% and a second row of one, two or three).
%
% The command \alignauthor (no curly braces needed) should
% precede each author name, affiliation/snail-mail address and
% e-mail address. Additionally, tag each line of
% affiliation/address with \affaddr, and tag the
% e-mail address with \email.
%
% 1st. author
\alignauthor 
Christan Grant\\
\affaddr{Dept. of Computer Science}\\
\affaddr{University of Florida}\\
\affaddr{Gainesville, Florida, USA}\\
\email{cgrant@cise.ufl.edu}
% 2nd. author
\alignauthor 
Clint P. George\\
\affaddr{Dept. of Computer Science}\\
\affaddr{University of Florida}\\    
\affaddr{Gainesville, Florida, USA}\\
\email{cgeorge@cise.ufl.edu}
% 3rd. author
\alignauthor 
Joir-dan Gumbs\\
\affaddr{Dept. of Computer Science}\\
       \affaddr{University of Florida}\\
       \affaddr{Gainesville, Florida, USA}\\
       \email{jgumbs@cise.ufl.edu}
\and  % use '\and' if you need 'another row' of author names
% 4th. author
\alignauthor 
Joseph N. Wilson\\
\affaddr{Dept. of Computer Science}\\
       \affaddr{University of Florida}\\
       \affaddr{Gainesville, Florida, USA}\\
       \email{jnw@cise.ufl.edu}
% 5th. author
\alignauthor 
Peter J. Dobbins\\
\affaddr{Dept. of Computer Science}\\
       \affaddr{University of Florida}\\
       \affaddr{Gainesville, Florida, USA}\\
       \email{pjd@cise.ufl.edu}
}


\date{2 July 2010}


\maketitle

% Abstract 
\begin{abstract}
- It is difficult to answer questions that change over time
- However, if the method of answering the question can be realized, the question can be easily answered
- We are building a system that supports a learning how to answer particular questions
- We can answer questions that are semantically similar to our store of questions
- We do this by building a semantic library ...
\end{abstract}


% A category with the (minimum) three required fields
%\category{H.4}{Information Systems Applications}{Miscellaneous}

% A category including the fourth, optional field follows...
%\category{D.2.8}{Software Engineering}{Metrics}[complexity measures, performance measures]

\category{H.3.3}{Information Search and Retrieval}{Query formulation, Relevance feedback, Search process}
\category{H.3.4}{Systems and Software}{Question-answering (fact retrieval) systems}
\category{I.2.6}{Learning}{Concept learning}[Parameter Learning]
\category{I.2.7}{Natural Language Processing}{Language parsing and understanding, Text analysis}


\terms{Theory}

\keywords{Deep web, web data mining, ontology creation, NLP, semantic query matching, question answering}

% Introduction 
\section{Introduction}

% Outline 
% ---------------------

% 1. Motivation 
% 2. What are we proposing?  
% 3. Challenges
% 		3.1 User query processing 
% 		3.1 Need of an ontology 
% 4. Realm based question answering 
% 5. About the paper structure 

When traveling though a jungle to a destination, it is easy to get lost.  The first person to journey somewhere may make a number of mistakes when trying to find the best path to their destination. Those who come later find it easier to reach the destination if a well-marked trail has been created. Olsen and Malizia describe this idea as \emph{exploiting trails} \cite{5379671}.  Rather than treating a user's discovery experience as a unique entity, one can exploit the fact that a similar search may have already been performed.  In one study, almost 40 percent of all queries were repetitions of previous queries\cite{1277770}. Thus, reuse of prior searches is one way to optimize the search
process.  Morpheus is a question answering system motivated by reuse of prior web search pathways to yield an answer to a user query. Morpheus follows path-finders to their destinations and not only marks the trail, but also provides a taxi service to take followers to similar
destinations.

There are two distinct Morpheus user roles. A
\textit{path-finder} enters queries in the Morpheus web interface and
searches for an answer to the query using an instrumented web browser. 
This web tracking tool stores the query
and necessary information to revisit the pathways to the page where the path-finder 
found the answer. A \textit{path-follower} uses the Morpheus system much like a regular search engine with a natural language interface. The path-follower enters a question in a text box and receives a guided path to the answer. The system exploits previously found paths to provide an answer.

Morpheus represents user-query details in a semi-structured query (SSQ). It assumes the terms belong to classes of a \textit{consistent} realm-based ontology, that is, one having a singly rooted heterarchy whose subclass/superclass relations have meaningful semantic interpretations. When a path-follower enters a query, Morpheus ranks SSQs in the store based on class similarity. Suppose a regular user asks -\textit{ a 1997 Toyota Camry V6 needs what size tires?} In this query the classes associated with terms, e.g. \emph{Manufacturer} with \emph{Toyota}, helps us identify similar queries.


This paper discusses related question answering and ontology generation systems in section \ref{sec:relatedwork}. Section \ref{sec:systemarch} explains the Morpheus system and its implementation. In Section \ref{sec:results} we describe the current results of our approach.  Finally, we conclude with future goals for the system.




 

\section{Related Work}

1. QA systems 
2. Ontology generators 
3. NLP interfaces 


\section{System Architecure}

SSQ % Both


\subsection{Query Processing}


  \subsubsection{QRR} % Guide user  

  \subsubsection{NLP interface} % Regular usr 


\subsection{Query Matching}

  
\subsection{Quer Resolution Methods} 



\section{Results}


% 3-4 queries 
% Human made ontology 
% Topic Models 


\subsection{Discussion}


% 2. Current stage of the project 
% 3. Expected contribution to the topics of interest supported by iiWAS2010 

\section{Conclusion}



\section{Acknowledgments}
Our thanks to Ben Landers, Patrick, and Terrence Thai for their assistance building this system. Additionally, we would like to thank NSF for their generous support of our project.

% Bibliography 
\bibliographystyle{abbrv}
\bibliography{morpheus}  


\end{document}
